Reservoir modeling is a well-utilized tool in chemical geodynamics to model the evolution of trace elements and radiogenic isotopes though a set of reservoirs~\cite{ACJ-BO-DB:1980, KLH-WGJ:1990, SNH-ZK:2001}. We employ reservoir modeling to model the evolution of deep carbon from post giant moon-forming impact to present time. However, due to the uncertainty in abundances of carbon within solid Earth, we constuct simple models to understand the control parameters of the carbon cycle. The undertainty history requires flexibilty in our models to test different concentrations of carbon, differnt rates of fluxes of carbon from reservoir to reservoir. Using python in conjunction with Jupyter notebooks, I construct interactive deep carbon reservoir models with the goal of being extensible. 

In creating this model, we identify the following reservoirs of carbon on Earth: the atmosphere, continental crust, and mantle. The atmosphere has been proposed as a key proponent in Earth's early history in light of the carbon concentration in Venusian atmosphere. We consider degassing from mantle to atmosphere due to island arc volcanism and mid-ocean ridge basalt (MORB) volcanism, drawdown of carbon from the atmosphere due to Urey reaction, and transport from continental crust to mantle as a result of subduction. We then investigate the source of carbon in the continental crust in three different cases. In case one, the carbon in the continental crust was built up by the drawdown of carbon from the atmosphere via Urey reaction only. In case two, the carbon in the continental crust was built due to a higher flux of arc volcanism relative to the rate of subduction. In case three, the carbon in the continental crust is a combination of the processes in case one and case two.

In developing these various models, we find similarities with other proposed reservoir models for different trace elements and/or radiogenic isotopes. While building the carbon models, at each step, I generalize to the arbitrary case arguing for the possibilty of a self-contained code. This constructed rservoir modeling framework in light of deep carbon is then scrutinized and accessed.
