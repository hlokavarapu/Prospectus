Reservoir modeling is a well-utilized tool in chemical geodynamics to model the evolution of trace elements and radiogenic isotopes~\cite{ACJ-BO-DB:1980, KLH-WGJ:1990, SNH-ZK:2001}. We build a reservoir model for the evolution of deep carbon from post moon-forming impact to present time. The key reservoirs of carbon in Earth are the continental crust, mantle, and the atmosphere. The atmosphere has been proposed as a key proponent in Earth's early history in light of the carbon concentration in Venusian atmosphere. We consider degassing from mantle to atmosphere due to mid-ocean ridge basalt (MORB) volcanism, ingassing of carbon from continental crust to mantle due to silicate weathering and subduction. We then investigate the source of carbon in the continental crust in three different cases. In case one, the carbon in the continental crust was built up by the drawdown of carbon from the atmosphere via Urey reaction only. In case two, the carbon in the continental crust was built due to a higher flux of volcanism relative to the rate of subduction. In case three, the carbon in the continental crust is a combination of the processes in case one and case two.

In developing these various models, we find similarities with other proposed reservoir models for different trace elements and/or radiogenic isotopes. While building the carbon models, at each step, I generalize to the arbitrary case arguing for the possibilty of a self-contained code. This constructed rservoir modeling framework in light of deep carbon is then scrutinized and accessed.