Recent measurements of carbon abundace in the atmosphere is $8.5 \times 10^2$ Gt~\cite{NOAA:2017} consistent with an increasing trend primarily attributed to anthropogenic influences~\cite{DDJ:2015}. Earth's natural carbon cycle transports carbon from the atmosphere into the oceans by the Urey reaction where both organic and inorganic percipiation of carbonates occurs~\cite{UHC:1952}. The fate of these carbonates varies but one such possibility is the drawndown of carbonates into the mantle at subduction zones. As carbonates are moving along the subducting slab, a portion of the carbon may be degassed back into the atmosphere as a result of island arc volcanism. The remainder is transported into the mantle, the second most abundant reservoir of carbon. In contrast to subduction, carbon is degassed from the mantle into the atmosphere as a result of mid-ocean ridge volcanism. The uncertainties in current carbon abundance estimates in Earth's reservoirs, and the uncertainties in the carbon flux rates between Earth's reservoirs requires constructing simple reservoir models. Reservoir modeling is a well-utilized tool in chemical geodynamics to model the evolution of trace elements and radiogenic isotopes~\cite{ACJ-BO-DB:1980}. By parameterizing the differnt processes such as the Urey reaction, MORB volcanism, or island-arc volcanism, we hope to study the implications of variable rates on the time evolution of carbon in the system as a whole.



The complexities in quantifying the carbon fluxes for different mchansisms, and carbon abundances in the different reservoirs, makes this a difficult problem. As a result, we employ reservoir modeling, a well-utilized tool in chemical geodynamics to model the evolution of trace elements and radiogenic isotopes ~\cite{ACJ-BO-DB:1980, KLH-WGJ:1990, SNH-ZK:2001}. By constructing simple reservoir models we hope to quantitatively study the influence of variations in MORB volcanism, island arc volcanism, rate of Urey reaction and its consequences on the complete system.


of the Earth is well known in the atmosphere but is porrly cnstrained in the solid Earth. Understanding the evolution of carbon over the 4.4 Ga adds to the level of uncertainty. As a result, we construct simple carbon reservoir models to model the evolution of carbon through the following reservoirs: atmosphere, continental crust, and mantle. ~\citet{SNH-ZK:2001} proposed that the Venusian atmosphere is analogus to Earth's earth atmospheric histoy consisting of $1.57 \times 10^8$ Gt of carbon. We parametrize the rate of silicate weathering which transfers carbon from the atmosphere and into the mantle effectively. 


Reservoir modeling is a well-utilized tool in chemical geodynamics to model the evolution of trace elements and radiogenic isotopes though a set of reservoirs~\cite{ACJ-BO-DB:1980, KLH-WGJ:1990, SNH-ZK:2001}. We employ reservoir modeling to model the evolution of deep carbon from post giant moon-forming impact to present time. However, due to the uncertainty in abundances of carbon within solid Earth, we constuct simple models to understand the control parameters of the carbon cycle. The undertainty history requires flexibilty in our models to test different concentrations of carbon, differnt rates of fluxes of carbon from reservoir to reservoir. Using python in conjunction with Jupyter notebooks, I construct interactive deep carbon reservoir models with the goal of being extensible. 

In creating this model, we identigy the following reservoirs of carbon on Earth: the atmosphere, continental crust, and mantle. The atmosphere has been proposed as a key proponent in Earth's early history in light of the carbon concentration in Venusian atmosphere. We consider degassing from mantle to atmosphere due to mid-ocean ridge basalt (MORB) volcanism, ingassing of carbon from continental crust to mantle due to silicate weathering and subduction. We then investigate the source of carbon in the continental crust in three different cases. In case one, the carbon in the continental crust was built up by the drawdown of carbon from the atmosphere via Urey reaction only. In case two, the carbon in the continental crust was built due to a higher flux of volcanism relative to the rate of subduction. In case three, the carbon in the continental crust is a combination of the processes in case one and case two.

In developing these various models, we find similarities with other proposed reservoir models for different trace elements and/or radiogenic isotopes. While building the carbon models, at each step, I generalize to the arbitrary case arguing for the possibilty of a self-contained code. This constructed rservoir modeling framework in light of deep carbon is then scrutinized and accessed.
