%In reservoir models, the mantle is often considered as a homogeneous reservoir however OIB and MORB samples show distinct groups plotted in $\eta_{Nd}$ versus $^{87}Sr/^{86}Sr$ ~\cite{WWM:2015}. Due to the volumetric differences on OIB and MORB volcanism, I propose subdividing the mantle into two reservoirs.
\subsection{Source of Carbon in the Continental Crust}
For the defined reservoir model shown in Fig.~\ref{Fig:ReservoirFlowDiagram}, the present abundances of carbon in atmosphere, continental crust, and mantle can be realized from either evolution curves shown in Figs.~\ref{Fig:ModelM-CC} or~\ref{Fig:ModelA-CC}. In model A-CC, from the giant-moon forming impact to the time of initial crustal formation, the mantle removes carbon from the atmosphere directly within the time period of $\tau_{am}$. From the initial time of crustal formation to present day, remaining carbon in the atmosphere is transferred to the continental crust as a result of Urey reaction. In model M-CC, the amount of carbon in the atmosphere is assumed to be similar in abundance to present day. If a higher rate of carbon was degassed as a result of volcanism in comparison to subduction of carbon over geological time, the source of carbon in the continental crust can be the mantle. By introducing an additional parameter, we can combine both models to reproduce the present modern abundances.

\subsection{Reservoir Modeling Framework}
In combination with object oriented software design, an arbitrary number of reservoirs, elements per reservoir, pathways, and complex flux functions would be possible. Once a reservoir model is constructed as a directed graph representation and initial conditions are specified, assembly and solution of the underlying ODE system can be solved using available numerical packages within Python. In order to discover edge cases in the proposed reservoir modeling framework, I plan to benchmark the code extensively after initially  reproducing results shown in Fig.~\ref{Fig:ModelM-CC} and~\ref{Fig:ModelA-CC}. 

A combination of the two models will require additional logic by linking both the initial conditions of the mantle reservoir and the atmosphere reservoir and linking the flux of carbon due to the Urey reaction and the ratio of transport due to subduction and mid-ocean ridge volcanism. The complexity will be a useful test of the proposed framework. 

Another potential candidate is the reservoir model of $^{3}\mathrm{He}/^{4}\mathrm{He}$ transport proposed by~\citet{KLH-WGJ:1990}. This model adds the complexity of a production rate of $^{4}\mathrm{He}$ due to the decay of multiple species, $^{235}\mathrm{U}$, $^{238}\mathrm{U}$, and $^{232}\mathrm{Th}$. In order to treat multiple radiogenic species, the time evolution of a reservoir  will require additional terms. Such a model will further test the applicability of proposed framework to a more generalized problem than the carbon models.

A second more complicated candidate is the proposed reservoir model in \cite{KJB-JSB-ORJ:2002}. Similarly to $^{3}\mathrm{He}/^{4}\mathrm{He}$ transport proposed by~\citet{KLH-WGJ:1990}, this model contains multiple radiogenic species as well. They introduce additional subreservoirs in the mantle as continental crust is generated and recycled. Dynamically adding subreservoirs through time will be challenging and further test the proposed framework.

A flexible reservoir modeling toolkit is necessary if this toolkit is to be utilized for a wide range of applications including different element cycles, radiogenic isotopes, and transport through porous media. Due to the wide applications of reservoir modeling and unforeseen special cases, the toolkit must be extensible. Extensibility will require transparency of the code in terms of readability and documentation for the library to be successful.

\subsection{Milestones on the Path to Master's Degree}
Presently, I am on track to complete my course requirements within the Fall quarter of 2018. In terms of my research goals within the time frame of this summer, I hope to have a working prototype of the proposed reservoir modeling framework. To achieve this goal in a timely fashion, there are three additional intermediary milestones. The first milestone will be a tested working implementation of the underlying data structures including directed graph and flux vectors proposed in Sec.~\ref{Sec:Methods}. A second milestone is then the rewriting of carbon Models A-CC, M-CC, and A/M-CC in the proposed framework. Lastly, by the end of the summer, I plan on implementing the other discussed benchmarks and adding them as example cookbooks. At this point, I would be pleased to release a beta version of the software in which others can test and utilize the reservoir modeling library in the hopes of obtaining constructive feedback. After further discussion and meeting said concerns, I hope that the software will smoothly transition to an initial release version. In the short term, this is my proposed path.
