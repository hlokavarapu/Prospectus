Earth's core is considered to contain a large fraction of Earth's carbon inventory. The source of the carbon is attributed to core crystallization and core amalgamation from the giant moon-forming impact. ~\cite{DR-CH-SN:2013} estimates the mass of carbon in the core is $4.44 \times 10^9$ Gt ($1~\text{Gt} = 10^{12}$ Kg). Since the giant moon-forming impact, there is little to no evidence of the exchange of carbon from the core to the mantle. 

In reservoir models, the mantle is often considered as a homogeneous reservoir however OIB and MORB samples show distinct groups plotted in $\eta_{Nd}$ versus $^{87}Sr/^{86}Sr$ ~\cite{WWM:2015}. Other reservoir models, differentiate the reservoir into a depleted mantle and an enriched mantle due to extraction from the formation of continental crust ~\cite{KJB-JSB-ORJ:2002}. 

~\citet{KWH:1995} estimates $4.2 \times 10^7$ Gt of carbon in the continental crust attributing about 80\% to carbonate sediments. ~\citet{UHC:1952} attributed the source of this carbon as a result of drawdown from the atmosphere due to exposed $\mathrm{Ca}^{2+}$ or $\mathrm{Mg}^{2+}$ silicate bearing rocks undergoing silicate weathering in a reaction known as Urey reaction~\cite{UHC:1952}

\begin{equation}
\label{EQ:Urey_reaction}
  \text{CO}_2 + \text{CaSiO}_3 \rightarrow \text{CaCO}_3 + \text{SiO}_2.
\end{equation}

\noindent The controlling parameters of this reaction are poorly understood but show dependence on the concentration of C, temperature in the atmosphere, exposed surface area of continental crust, and rainfall.  

Although the atmosphere today relatively contains a small fraction of Earth's carbon inventory,~\citet{KJD:2002} suggests that a significant fraction of Earth;s initial carbon inventory resided in the atmosphere which was subsequently drawn down until cooling of global magma ocean and then further drawn down as the continental crust was being built. ~\citet{KHD:2002} estimates that this drawdown before reaching quasi-steady state could of been completed by the end of Archean or 1.5Ga ago at the latest. Alternatively, the atmosphere may have contained only a small fraction of the Earth;s carbon inventory thoughout its history with the Urey reaction as an important controlling process on the flux rate of carbon from the atmosphere to the continental crust. The source of the carbon in this model is associated with the mantle due to a higher flux of carbon release from volcanism in comparison to the flux of carbon subducted.

The oceans are an extremely important reservoir although they contain a relatively insignificant amount of carbon. The byproducts of silicate weathering are transported into the oceans by river systems where organic and inorganic percipitation of carbonates occurs. The resulting pelagic carbonates at the oceanic floor is then subducted or resurfaced as carbon dioxide byproduct of reverse Urey reaction as island arc volcanism. MORB volcanism and hydrothermal vents degass carbon whose source is the mantle. Thus, the ocean plays and important role acting as an intermediary reservoir in the long term carbon cycle.

Due to the numerous processes associated with the deep carbon cycle, precise quantitaive measurements over the lifetime of Earth are lacking. Therefore, simpliyfing the model to the basic set of reservoirs and/or fluxes is vital otherwise the results maybe meaningless. ~\citet{SNH-ZK:2001} constructed a simple carbon reservoir model that spanned from the Archean to present day. Their reservoir model consisted of the following reservoirs: continental sediments, ocean, oceanic crust, and mantle. 