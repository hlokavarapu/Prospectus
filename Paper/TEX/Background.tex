High pressure experiments indicate that carbon is a siderophile element with a very large alloy/silicate partition coefficient ~\cite{DR-CH-SN:2013}. The crystallization of the core and the moon-forming impact resulted in the partitioning of 95\% of Earth's carbon inventory. Since the giant moon-forming impact, there is little to no evidence of the exchange of carbon from the core to the mantle. Estimates for the abundance of carbon in the core is based on a chondritic model. The uncertainty in estimates of carbon abundance in core encapsulates the sum of carbon in all other reservoirs presently.

In reservoir models, the mantle is often considered as a homogenous reservoir however OIB and MORB samples show distinct groups plotted in $\eta_{Nd}$ versus $^{87}Sr/^{86}Sr$.

~\citet{SNH-ZK:2001} constructed a simple carbon reservoir model that spanned from the Archean to present day. Their reservoir model consisted of the following reservoirs: continental sediments, ocean, oceanic crust, and mantle. 

{\renewcommand{\arraystretch}{2.0}% for the vertical padding
\begin{table}[b!]
    \centering
    \begin{tabular}{|l|l|c|c|}
        \hline
        Reservoir & Mass of carbon Gt & Reference \\
        \hline
        Core   & $4 \times 10^9$ & ~\citet{DR:2013}  \\
        \hline
        Mantle & $2 \times 10^8$ & ~\citet{KLH-TDL-WM:2017}  \\
        \hline
        Continental crust & $4.2 \times 10^7$ & ~\citet{KHW:1995} \\
        \hline
        Oceans & $3.8 \times 10^4$ & ~\citet{HRA:2007} \\
        \hline
        Atmosphere & $8.5 \times 10^2$ & ~\citet{NOAA:2017} \\
        \hline
        Total & $4.24 \times 10^9$ & \\
        \hline
    \end{tabular}
    \caption{\doublespacing Masses of carbon in the Earth's carbon reservoirs considered in this paper (1 Gt $= 10^{12}$ kg).
    }
    \label{Table:Masses of carbon in Earth's reservoirs}
\end{table}
