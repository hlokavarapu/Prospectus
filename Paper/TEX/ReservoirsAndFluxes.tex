Carbon has been partitioned into five main reservoirs: the core, the mantle, the continental crust, the oceans, and the atmosphere.

\subsection{Core}
	\begin{enumerate}
		\item Carbon is a siderophile element with a very large alloy/silicate partition coefficient~\cite{DR-CH-SN:2013}.
	\end{enumerate}
\subsection{Mantle}
	\begin{enumerate}
		\item Metamorphism, reverse Urey reaction
		\item Melting/Volcanism
		\item Subduction
		\item Removal of Carbon due to large alloy/silicate partition coefficient
	\end{enumerate}
\subsection{Continental Crust}
	\begin{enumerate}
		\item Weathering of exposed calcium bearing silicate rocks from acidic rain with dissolved carbon results in an outflux of calcium cations, silicates, carbonic acid and other chemicals transported by river systems into the oceans. This process is driven by the Urey reaction.
	\end{enumerate}
\subsection{Oceans}
	\begin{enumerate}
		\item Silicate weathering generates an influx of calcium and carbonic acid.
		\item Hydrothermal vents play a role in the percipitation of inorganic carbonates.
		\item Formanifera, diatoms life cycle results in deposition of calcite onto the ocean floor.
		\item Mid-Ocean ridge volcanism
		\item Subduction processes
		\item Serpentenization
	\end{enumerate}
\subsection{Atmosphere}
	\begin{enumerate}
		\item Carbon is comparatively insoluble with respect to other volatiles~\cite{HMM:2016}.
	\end{enumerate}