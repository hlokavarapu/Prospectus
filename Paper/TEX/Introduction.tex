Although carbon has become a polarized topic in light of global warming, uncertainties are countless. The carbon cycle is a complicated system involving chemical weathering of silicate rocks, serpentenization at ocean floor subduction zones, carbon gas emissions due to volanism, the complexities of mantle convection, Milankovich cycles, impact erosion to only name a few. To begin understanding these complexities we invoke a simple tool well utilized by the geochemists, reservoir modeling. The goal is to gain a first-order approximation by identifying the key reservoirs of carbon and the key fluxes of carbon. 


Reservoir modeling requires identification of the key reservoirs of carbon in terms of abundance and identification of the set of processes that move carbon from one reservoir to another.The key reservoirs that have been identified can be broadley broken down into the reservoirs of the deep interior and surficial reservoirs. Reservoirs of the deep interior include the Earth's core, mantle, and continental crust. Surficial reservoirs consist of the oceans and atmosphere.
