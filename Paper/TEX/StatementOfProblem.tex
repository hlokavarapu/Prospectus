At present day, 99\% of carbon is predominantly stored in the continental crust, core and mantle. The abundance of carbon in the surficial reservoirs such as the atmosphere and oceans is relatively minimal. However,~\citet{KJF-ATP:1986} suggest the possibility of a high abundance of carbon in the atmosphere during Earth's early history. Scaling the carbon concentration presently in Venus's atmosphere to Earth's atmosphere gives a total carbon mass of $^cM_{a} = 1.28 \times 10^8$ Gt. ~\citet{KJD:2002} suggests that this carbon mass has been fractionated into the mantle and the continental crust by the Archean at the earliest and 1.5 Ga at the latest. An alternative source of carbon in the continental crust may have been the mantle as a result of a higher flux of arc volcanism relative to subduction of oceanic crust into the deep interior. In order to test the possibility of either source of carbon in the continental crust, we construct a carbon reservoir model evolving from the moon-forming impact to present time. Preliminary results indicate that either source of carbon in the continental crust is possible. In particular, the models show the time evolution to the present partitioning of carbon is possible within the constraints posited for either case. In the future, I propose testing the viability of a combination of both sources to build the carbon content in the continental crust.

Reservoir modeling is a well-utilized tool in chemical geodynamics. However, there is a lack of transparent, flexible and generalized open-source reservoir modeling code. In order to fill this gap, I propose to build a python based library consisting of the following modules: reservoirs, fluxes, assembly and solution of ODE system, and visualization of results. The reservoir module will consist of an associated mass for the single element or isotope concentration. The fluxes consist of a source function, two end member reservoirs, and direction of transport. This allows for an arbitrary number of reservoirs and fluxes to be specified. Once the model has been constructed, the resulting ODE system will need to be assembled and solved for. Lastly, a user-friendly visualization toolkit to help analyze the results is important whether that includes plotting or mere access to the raw data. In order to verify the viability of the proposed framework, one idea is to implement other established reservoir models as examples.